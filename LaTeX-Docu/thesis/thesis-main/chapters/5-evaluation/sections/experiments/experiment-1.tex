% The subsection about the results of the first experiment
% @author Kalvin Döge
%

\subsection{Experiment 1: 70 g | 20 r}\label{subsec:results}

Im ersten Experiment werden die Zeiten der Obergrenze \code{35 g | 40 r} genommen und die Dauer der Rotphase halbiert und die der Grünphase verdoppelt: \code{70 g | 20 r}.
Mit den neuen Zeiten ergeben die Experimente folgende Ausgangswerte:

\begin{table}[htb]
    \centering
    \begin{tabular}{||c|c|c|c|c|c|c|c|c|c|c|c||}
        \hline
        Stunde  & \#NHT  & E1  & E2  & E3  & E4  & E5  & E6  & E7  & E8  & E9  & E10 \\\hline\hline
        0       & 14     & \qg & \qr & \qr & \qr & \qr & \qr & \qg & \qr & \qg & \qr \\\hline
        1       & 51     & \qr & \qg & \qg & \qr & \qg & \qr & \qr & \qg & \qr & \qg \\\hline
        2       & 130    & \qr & \qg & \qr & \qr & \qr & \qr & \qr & \qg & \qr & \qr \\\hline
        3       & 274    & \qg & \qg & \qg & \qg & \qg & \qr & \qg & \qr & \qg & \qr \\\hline
        4       & 487    & \qr & \qg & \qr & \qg & \qr & \qr & \qr & \qr & \qg & \qg \\\hline
        5       & 707    & \qg & \qr & \qr & \qg & \qg & \qr & \qr & \qr & \qg & \qg \\\hline
        6       & 857    & \qr & \qr & \qr & \qr & \qr & \qg & \qg & \qr & \qr & \qr \\\hline
        7       & 879    & \qr & \qr & \qg & \qg & \qr & \qr & \qg & \qr & \qr & \qg \\\hline
        8       & 804    & \qg & \qr & \qg & \qg & \qr & \qr & \qg & \qr & \qr & \qg \\\hline
        9       & 684    & \qg & \qg & \qg & \qg & \qr & \qr & \qr & \qr & \qr & \qg \\\hline
        10      & 570    & \qg & \qr & \qr & \qg & \qg & \qg & \qg & \qr & \qr & \qg \\\hline
        11      & 514    & \qg & \qr & \qr & \qr & \qr & \qr & \qr & \qg & \qr & \qg \\\hline
        12      & 550    & \qr & \qg & \qr & \qr & \qr & \qg & \qg & \qr & \qg & \qr \\\hline
        13      & 650    & \qr & \qg & \qr & \qg & \qr & \qg & \qg & \qg & \qr & \qg \\\hline
        14      & 765    & \qr & \qr & \qr & \qg & \qr & \qg & \qr & \qg & \qg & \qr \\\hline
        15      & 850    & \qg & \qg & \qr & \qg & \qr & \qg & \qg & \qr & \qg & \qg \\\hline
        16      & 857    & \qg & \qr & \qr & \qg & \qg & \qg & \qr & \qr & \qg & \qr \\\hline
        17      & 755    & \qg & \qr & \qg & \qr & \qr & \qr & \qr & \qr & \qg & \qg \\\hline
        18      & 579    & \qr & \qg & \qr & \qr & \qg & \qr & \qr & \qr & \qr & \qr \\\hline
        19      & 379    & \qr & \qg & \qr & \qg & \qg & \qr & \qg & \qr & \qr & \qr \\\hline
        20      & 205    & \qg & \qg & \qg & \qg & \qr & \qg & \qg & \qr & \qr & \qg \\\hline
        21      & 94     & \qr & \qr & \qg & \qr & \qr & \qg & \qr & \qg & \qg & \qr \\\hline
        22      & 34     & \qg & \qr & \qg & \qr & \qr & \qg & \qr & \qr & \qg & \qr \\\hline
        23      & 14     & \qr & \qg & \qg & \qr & \qg & \qg & \qg & \qg & \qr & \qr \\\hline\hline
        Gesamt: & 11.703 & \qf & \qf & \qf & \qf & \qf & \qf & \qf & \qf & \qf & \qf
    \end{tabular}
    \caption{Erfolgstabelle des 1. Experimentes}
    \label{tab:experiment-1-table}
    \centering
\end{table}

Aus der Tabelle~\ref{tab:experiment-1-table} lässt sich entnehmen, dass von 240-Stunden-Simulationen 108 keine und 132 eine ,,grüne Welle`` abbekomnen haben.
Der Durchschnitt an fehlgeschlagenen Stundensimulationen beträgt ungefähr 11 Fehlversuche:
\[(12 + 12 + 10 + 13 + 8 + 11 + 12 + 7 + 11 + 12) / 10 = 10,8\]
Die Erfolgsrate von 90\% ist damit noch nicht erreicht mit diesen Lichtsignalzeiten.
Die erfolgreichen, ,,grünen Wellen`` sind der Tabelle~\ref{tab:experiment-1-table} nach noch beliebig verteilt und dabei weder von der Anzahl der neu hinzugefügten \code{HumanTraveler} noch von der Tageszeit beeinflusst.
Im Vergleich zu der Obergrenze ist aber eine erhöhte Menge an erfolgreichen, grünen Wellen zu sehen.
