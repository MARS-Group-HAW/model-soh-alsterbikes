% The subsection about the results of the third experiment
% @author Kalvin Döge
%

\subsection{Experiment 3: 280 g | 5 r}\label{subsec:experiment-3}

Das dritte Experiment halbiert aufgrund des Nicht-Erreichens der 90\% ebenfalls wieder die Rotphase und verdoppelt die Grünphasenlänge der vorangegangenen Schaltung von \code{140 g | 10 r} auf \code{280 g | 5 r}.
Mit den neuen Zeiten werden die Simulationen wieder ausgeführt und geben dabei diese Werte aus:

\begin{table}[htb]
    \centering
    \begin{tabular}{||c|c|c|c|c|c|c|c|c|c|c|c||}
        \hline
        Stunde  & \#NHT  & E1  & E2  & E3  & E4  & E5  & E6  & E7  & E8  & E9  & E10 \\\hline\hline
        0       & 14     & \qr & \qg & \qg & \qr & \qg & \qg & \qg & \qg & \qg & \qr \\\hline
        1       & 51     & \qg & \qr & \qg & \qg & \qg & \qg & \qg & \qg & \qg & \qr \\\hline
        2       & 130    & \qg & \qg & \qg & \qg & \qg & \qg & \qg & \qr & \qg & \qg \\\hline
        3       & 274    & \qg & \qg & \qg & \qg & \qr & \qg & \qg & \qg & \qr & \qg \\\hline
        4       & 487    & \qg & \qr & \qg & \qg & \qg & \qg & \qr & \qg & \qg & \qg \\\hline
        5       & 707    & \qg & \qg & \qr & \qr & \qr & \qr & \qg & \qg & \qg & \qg \\\hline
        6       & 857    & \qg & \qg & \qr & \qg & \qg & \qg & \qg & \qg & \qg & \qg \\\hline
        7       & 879    & \qg & \qg & \qg & \qg & \qg & \qg & \qg & \qg & \qg & \qg \\\hline
        8       & 804    & \qg & \qg & \qg & \qg & \qg & \qg & \qg & \qg & \qg & \qg \\\hline
        9       & 684    & \qg & \qg & \qg & \qg & \qg & \qg & \qg & \qg & \qg & \qg \\\hline
        10      & 570    & \qg & \qg & \qg & \qg & \qg & \qg & \qg & \qr & \qr & \qr \\\hline
        11      & 514    & \qr & \qg & \qg & \qg & \qg & \qg & \qg & \qg & \qg & \qg \\\hline
        12      & 550    & \qg & \qg & \qg & \qr & \qr & \qr & \qg & \qg & \qg & \qg \\\hline
        13      & 650    & \qg & \qg & \qg & \qg & \qg & \qg & \qg & \qr & \qg & \qg \\\hline
        14      & 765    & \qg & \qg & \qg & \qg & \qg & \qg & \qg & \qg & \qg & \qg \\\hline
        15      & 850    & \qg & \qg & \qg & \qg & \qg & \qg & \qg & \qg & \qg & \qg \\\hline
        16      & 857    & \qg & \qg & \qg & \qg & \qg & \qg & \qg & \qg & \qr & \qg \\\hline
        17      & 755    & \qg & \qg & \qg & \qg & \qg & \qg & \qr & \qg & \qg & \qg \\\hline
        18      & 579    & \qg & \qr & \qg & \qg & \qg & \qg & \qg & \qg & \qg & \qr \\\hline
        19      & 379    & \qg & \qg & \qg & \qg & \qg & \qg & \qg & \qg & \qg & \qg \\\hline
        20      & 205    & \qg & \qg & \qg & \qg & \qg & \qg & \qg & \qg & \qg & \qg \\\hline
        21      & 94     & \qg & \qg & \qg & \qr & \qg & \qg & \qg & \qg & \qg & \qg \\\hline
        22      & 34     & \qr & \qg & \qr & \qg & \qg & \qg & \qg & \qg & \qg & \qg \\\hline
        23      & 14     & \qg & \qg & \qg & \qr & \qg & \qg & \qg & \qg & \qg & \qg \\\hline\hline
        Gesamt: & 11.703 & \qf & \qf & \qf & \qf & \qf & \qs & \qs & \qf & \qf & \qf
    \end{tabular}
    \caption{Erfolgstabelle des 3. Experimentes}
    \label{tab:experiment-3-table}
    \centering
\end{table}

Aus der Erfolgstabelle~\ref{tab:experiment-3-table} lässt sich ablesen, dass von 240 Stundensimulationen 31 keine ,,grüne Welle`` und 209 eine ,,grüne Welle`` simulieren konnten.
Die Durchschnittsanzahl an fehlgeschlagenen Simulationen beträgt ungefähr 3:
\[(3 + 3 + 3 + 5 + 3 + 2 + 2 + 3 + 3 + 4) / 10 = 3,1\]
Weiterhin scheint die Verteilung der fehlgeschlagenen Simulationen über die Tabelle~\ref{tab:experiment-3-table} zufällig verteilt zu sein, während mehr als 2 Einträge im Durchschnitt pro Simulation noch immer fehlschlagen und damit die 90\% Erfolgswahrscheinlichkeit nicht erfüllen.
