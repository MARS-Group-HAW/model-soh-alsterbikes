% The subsection about the second experiment
% @author Kalvin Döge
%

\subsection{Experiment 2: 140 g | 10 r}\label{subsec:experiment-2}

In dem zweiten Experiment werden die Zeiten aus dem vorherigen Experiment \code{70 g | 20 r} genommen und erneut verdoppelt sowie halbiert: \code{140 g | 10 r}.
Die neu berechneten Zeiten haben dann den folgenden Effekt auf die Experimente:

\begin{table}[htb]
    \centering
    \begin{tabular}{||c|c|c|c|c|c|c|c|c|c|c|c||}
        \hline
        Stunde  & \#NHT  & E1  & E2  & E3  & E4  & E5  & E6  & E7  & E8  & E9  & E10 \\\hline\hline
        0       & 14     & \qr & \qg & \qg & \qg & \qr & \qg & \qr & \qg & \qg & \qg \\\hline
        1       & 51     & \qg & \qr & \qg & \qg & \qg & \qg & \qg & \qr & \qg & \qg \\\hline
        2       & 130    & \qg & \qr & \qr & \qg & \qg & \qg & \qg & \qg & \qr & \qg \\\hline
        3       & 274    & \qg & \qg & \qg & \qg & \qg & \qr & \qg & \qg & \qg & \qr \\\hline
        4       & 487    & \qg & \qg & \qg & \qg & \qr & \qr & \qr & \qg & \qg & \qg \\\hline
        5       & 707    & \qg & \qg & \qr & \qg & \qg & \qg & \qr & \qr & \qr & \qg \\\hline
        6       & 857    & \qg & \qr & \qg & \qg & \qg & \qg & \qg & \qg & \qr & \qg \\\hline
        7       & 879    & \qr & \qg & \qg & \qg & \qg & \qg & \qg & \qr & \qg & \qr \\\hline
        8       & 804    & \qg & \qr & \qg & \qg & \qg & \qg & \qg & \qg & \qg & \qg \\\hline
        9       & 684    & \qg & \qg & \qr & \qr & \qg & \qr & \qg & \qg & \qg & \qg \\\hline
        10      & 570    & \qr & \qg & \qg & \qg & \qg & \qg & \qg & \qg & \qg & \qr \\\hline
        11      & 514    & \qg & \qg & \qg & \qg & \qg & \qg & \qg & \qg & \qr & \qg \\\hline
        12      & 550    & \qg & \qg & \qg & \qr & \qr & \qg & \qg & \qg & \qr & \qg \\\hline
        13      & 650    & \qr & \qg & \qg & \qg & \qg & \qg & \qg & \qg & \qg & \qg \\\hline
        14      & 765    & \qg & \qg & \qg & \qg & \qg & \qg & \qg & \qg & \qg & \qr \\\hline
        15      & 850    & \qg & \qg & \qg & \qr & \qg & \qg & \qg & \qg & \qr & \qg \\\hline
        16      & 857    & \qg & \qg & \qg & \qg & \qr & \qg & \qg & \qg & \qg & \qr \\\hline
        17      & 755    & \qg & \qg & \qg & \qg & \qg & \qr & \qg & \qg & \qg & \qr \\\hline
        18      & 579    & \qg & \qg & \qr & \qg & \qg & \qr & \qg & \qg & \qg & \qg \\\hline
        19      & 379    & \qg & \qr & \qg & \qg & \qg & \qg & \qg & \qg & \qg & \qr \\\hline
        20      & 205    & \qg & \qg & \qg & \qg & \qg & \qg & \qg & \qr & \qg & \qg \\\hline
        21      & 94     & \qg & \qg & \qg & \qg & \qr & \qg & \qg & \qg & \qr & \qg \\\hline
        22      & 34     & \qg & \qr & \qg & \qg & \qg & \qg & \qr & \qg & \qg & \qg \\\hline
        23      & 14     & \qg & \qg & \qr & \qr & \qr & \qr & \qg & \qg & \qg & \qg \\\hline\hline
        Gesamt: & 11.703 & \qf & \qf & \qf & \qf & \qf & \qf & \qf & \qf & \qf & \qf
    \end{tabular}
    \caption{Erfolgstabelle des 2. Experimentes}
    \label{tab:experiment-2-table}
    \centering
\end{table}

Die Tabelle~\ref{tab:experiment-2-table} zeigt, dass bei 240-Stunden-Simulationen 53 keinen Erfolg und 187 Erfolg hatten, eine ,,grüne Welle`` zu simulieren.
Durchschnittlich schlugen dabei 5 Simulationen an einem Tag fehl:
\[(4 + 6 + 5 + 4 + 6 + 6 + 3 + 4 + 7 + 7) / 10 = 5,2\]
Weiterhin ist die Erfolgsrate von 90\% mit diesen Lichtsignalzeiten nicht erreicht, während dennoch die Anzahl der ,,grünen Wellen`` weiter ansteigt.
