% The subsection about the results of the upper and lower bound experiments
% @author Kalvin Döge
% 35g 40r up
% 35g  0r low
%

\subsection{Ober- und Untergrenze der Simulation}\label{subsec:lower-upper-bound}

Im Folgenden werden die Ober- und Untergrenzen der Simulation evaluiert.
Die Obergrenze stellt hierbei die Schaltung \code{35 g | 40 r} das dar, was der Realität angenähert ist.

Damit die Spalten übersichtlich bleiben, werden folgende Abkürzungen verwendet:
\begin{itemize}
    \item \code{Stunde} stellt hier die Stunde am Tag dar.
    \item \code{\#NHT} meint die Anzahl der zu dieser Stunde neu hinzugefügten \code{HumanTraveler} - ,,(N)eue (H)uman(T)raveler``.
    \item \code{E1, E2 ... E10} meint hier, ob das Experiment 1, 2 und so weiter erfolgreich war, eine grüne Welle für den BicycleLeader zu simulieren.~Ein grüner Hintergrund meint dabei Erfolg, ein roter Hintergrund kein Erfolg.
    \item \code{Gesamt} meint die Gesamtbilanz am Ende, also die Gesamtmenge an erstellten \code{HumanTraveler} und ob die 90\%-Erfolgsquote eingehalten wurde mit einem dunkleren, grünen Hintergrund oder nicht mit einem dunkleren, roten Hintergrund.
\end{itemize}

\begin{table}[htb]
    \centering
    \begin{tabular}{||c|c|c|c|c|c|c|c|c|c|c|c||}
        \hline
        Stunde  & \#NHT  & E1  & E2  & E3  & E4  & E5  & E6  & E7  & E8  & E9  & E10 \\\hline\hline
        0       & 14     & \qr & \qr & \qg & \qr & \qr & \qr & \qr & \qr & \qr & \qr \\\hline
        1       & 51     & \qr & \qr & \qr & \qr & \qr & \qr & \qr & \qg & \qr & \qr \\\hline
        2       & 130    & \qr & \qr & \qr & \qg & \qg & \qg & \qr & \qg & \qg & \qr \\\hline
        3       & 274    & \qr & \qg & \qr & \qg & \qr & \qg & \qr & \qr & \qr & \qr \\\hline
        4       & 487    & \qr & \qg & \qr & \qr & \qr & \qg & \qr & \qr & \qr & \qg \\\hline
        5       & 707    & \qr & \qr & \qg & \qr & \qr & \qr & \qr & \qr & \qg & \qr \\\hline
        6       & 857    & \qr & \qr & \qr & \qg & \qr & \qr & \qg & \qr & \qr & \qg \\\hline
        7       & 879    & \qg & \qr & \qg & \qr & \qr & \qr & \qr & \qr & \qr & \qr \\\hline
        8       & 804    & \qg & \qr & \qg & \qr & \qr & \qr & \qr & \qr & \qr & \qr \\\hline
        9       & 684    & \qr & \qr & \qr & \qr & \qr & \qr & \qr & \qr & \qg & \qr \\\hline
        10      & 570    & \qr & \qr & \qg & \qr & \qr & \qr & \qr & \qr & \qr & \qg \\\hline
        11      & 514    & \qg & \qr & \qr & \qr & \qr & \qg & \qg & \qr & \qr & \qr \\\hline
        12      & 550    & \qr & \qr & \qr & \qr & \qr & \qr & \qr & \qr & \qr & \qr \\\hline
        13      & 650    & \qr & \qr & \qr & \qr & \qr & \qr & \qr & \qr & \qr & \qr \\\hline
        14      & 765    & \qr & \qr & \qr & \qr & \qr & \qr & \qr & \qr & \qr & \qr \\\hline
        15      & 850    & \qr & \qr & \qr & \qr & \qr & \qr & \qr & \qr & \qr & \qr \\\hline
        16      & 857    & \qr & \qr & \qr & \qr & \qr & \qr & \qr & \qr & \qr & \qr \\\hline
        17      & 755    & \qr & \qr & \qr & \qr & \qr & \qr & \qr & \qr & \qr & \qr \\\hline
        18      & 579    & \qr & \qr & \qr & \qr & \qr & \qr & \qr & \qr & \qr & \qr \\\hline
        19      & 379    & \qr & \qr & \qr & \qr & \qr & \qr & \qr & \qr & \qr & \qr \\\hline
        20      & 205    & \qr & \qr & \qr & \qr & \qr & \qr & \qr & \qr & \qr & \qr \\\hline
        21      & 94     & \qr & \qr & \qr & \qr & \qr & \qr & \qr & \qr & \qr & \qr \\\hline
        22      & 34     & \qr & \qr & \qr & \qr & \qr & \qr & \qr & \qr & \qr & \qr \\\hline
        23      & 14     & \qr & \qr & \qr & \qr & \qr & \qr & \qr & \qr & \qr & \qr \\\hline\hline
        Gesamt: & 11.703 & \qf & \qf & \qf & \qf & \qf & \qf & \qf & \qf & \qf & \qf
    \end{tabular}
    \caption{Die Erfolgstabelle der Obergrenze für die Simulation mit 35 g | 40 r.}
    \label{tab:upper-bound-table}
    \centering
\end{table}

Wie zu erwarten ist die Erfolgschance des \code{BicycleLeader} sehr gering.
Von 240 Versuchen, eine ,,grüne Welle`` zu erreichen, schlugen 212 fehl.
Die durchschnittliche Anzahl an fehlgeschlagenen ,,grünen Wellen`` beträgt bei der Obergrenze 21,2:
\[(21 + 22 + 19 + 21 + 23 + 20 + 22 + 22 + 21 + 21) / 10 = 21,2 \]
Die Ergebnisse der Tabelle~\ref{tab:upper-bound-table} stellen damit die zu verbesserne Ausgangslage, wie sie für die nachfolgenden Experimente verbessert werden soll.
Im folgenden wird nun die Untergrenze angeführt, die es von den Phasenlängen \code{35 g | 0 r} dabei nicht zu unterschreiten gilt:

\begin{table}[htb]
    \centering
    \begin{tabular}{||c|c|c|c|c|c|c|c|c|c|c|c||}
        \hline
        Stunde  & \#NHT  & E1  & E2  & E3  & E4  & E5  & E6  & E7  & E8  & E9  & E10 \\\hline\hline
        0       & 14     & \qg & \qg & \qg & \qg & \qg & \qg & \qg & \qg & \qg & \qg \\\hline
        1       & 51     & \qg & \qg & \qg & \qg & \qg & \qg & \qg & \qg & \qg & \qg \\\hline
        2       & 130    & \qg & \qg & \qg & \qg & \qg & \qg & \qg & \qg & \qg & \qg \\\hline
        3       & 274    & \qg & \qg & \qg & \qg & \qg & \qg & \qg & \qg & \qg & \qg \\\hline
        4       & 487    & \qg & \qg & \qg & \qg & \qg & \qg & \qg & \qg & \qg & \qg \\\hline
        5       & 707    & \qg & \qg & \qg & \qg & \qg & \qg & \qg & \qg & \qg & \qg \\\hline
        6       & 857    & \qg & \qg & \qg & \qg & \qg & \qg & \qg & \qg & \qg & \qg \\\hline
        7       & 879    & \qg & \qg & \qg & \qg & \qg & \qg & \qg & \qg & \qg & \qg \\\hline
        8       & 804    & \qg & \qg & \qg & \qg & \qg & \qg & \qg & \qg & \qg & \qg \\\hline
        9       & 684    & \qg & \qg & \qg & \qg & \qg & \qg & \qg & \qg & \qg & \qg \\\hline
        10      & 570    & \qg & \qg & \qg & \qg & \qg & \qg & \qg & \qg & \qg & \qg \\\hline
        11      & 514    & \qg & \qg & \qg & \qg & \qg & \qg & \qg & \qg & \qg & \qg \\\hline
        12      & 550    & \qg & \qg & \qg & \qg & \qg & \qg & \qg & \qg & \qg & \qg \\\hline
        13      & 650    & \qg & \qg & \qg & \qg & \qg & \qg & \qg & \qg & \qg & \qg \\\hline
        14      & 765    & \qg & \qg & \qg & \qg & \qg & \qg & \qg & \qg & \qg & \qg \\\hline
        15      & 850    & \qg & \qg & \qg & \qg & \qg & \qg & \qg & \qg & \qg & \qg \\\hline
        16      & 857    & \qg & \qg & \qg & \qg & \qg & \qg & \qg & \qg & \qg & \qg \\\hline
        17      & 755    & \qg & \qg & \qg & \qg & \qg & \qg & \qg & \qg & \qg & \qg \\\hline
        18      & 579    & \qg & \qg & \qg & \qg & \qg & \qg & \qg & \qg & \qg & \qg \\\hline
        19      & 379    & \qg & \qg & \qg & \qg & \qg & \qg & \qg & \qg & \qg & \qg \\\hline
        20      & 205    & \qg & \qg & \qg & \qg & \qg & \qg & \qg & \qg & \qg & \qg \\\hline
        21      & 94     & \qg & \qg & \qg & \qg & \qg & \qg & \qg & \qg & \qg & \qg \\\hline
        22      & 34     & \qg & \qg & \qg & \qg & \qg & \qg & \qg & \qg & \qg & \qg \\\hline
        23      & 14     & \qg & \qg & \qg & \qg & \qg & \qg & \qg & \qg & \qg & \qg \\\hline\hline
        Gesamt: & 11.703 & \qs & \qs & \qs & \qs & \qs & \qs & \qs & \qs & \qs & \qs
    \end{tabular}
    \caption{Die Erfolgstabelle der Obergrenze für die Simulation mit 35 g | 0 r.}
    \label{tab:lower-bound-table}
    \centering
\end{table}

Wie sich logisch ableiten lässt, ist die Untergrenze mit \code{35 g | 0 r} stets funktionierend, wenn lediglich die ,,grüne Welle`` fokussiert wird.
Gibt es keine Rotphase, so tritt auch nicht die Abbruchbedingung des \code{BicycleLeader}s ein.
Da dies aber nicht der Realität entspricht und es bei Lichtsignalanlagen Rotphasen geben muss für die andere Seite der Kreuzung, soll eine Lichtsignalschaltung gefunden werden mit einer längeren Rotphase als 0 Sekunden.

Die nun folgenden Experimente sollen also mit der Antiproportionalität bei den Grün- und Rotphasen eine solche, ideale Schaltung annähern.
