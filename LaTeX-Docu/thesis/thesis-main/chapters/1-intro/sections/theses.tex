%
% @author Kalvin Döge
%
\section{Inhalt der Arbeit}\label{sec:theses}

Zur Bestimmung einer ,,Grünen Welle'' für Fahrradfahrer um die Binnen- und Außenalster, kommen folgende Fragen zur Simulation des Szenarios auf:

\begin{itemize}
    \item Wie sieht ein durchschnittlicher Fahrradfahrer, Fußgänger und ein Personenkraftwagen in dem Modell aus?
    \item Welche Strecke fährt der Fahrradfahrer um die Binnen- und Außenalster, um eine Rundfahrt unternommen zu haben?
    \item Was für Eigenschaften muss die Lichtsignalschaltung in dem Modell haben, damit sie geeignet für eine ,,Grüne Welle`` ist?
    \item Wie stark wirkt sich die Auslastung der Straßen auf das Lichtsignalnetz aus, wenn zu verschiedenen Uhrzeiten am Tag die Alsterrundfahrt unternommen wird?
    \item Wie müssen die Lichtsignalzeiten angepasst werden, wenn der Verkehr sich zu stark auf die ,,Grüne Welle'' auswirkt?
    \item Ist es überhaupt möglich, dass Fahrradfahrer in mindestens 90\% der Fälle, die er um die Alster fährt, eine ,,Grüne Welle'' erhält, ohne anzuhalten?
\end{itemize}

Um die Aspekte genauer zu untersuchen, lassen sich aus ihnen forschungsrelevante Hypothesen aufstellen, die im Folgenden genauer definiert werden:

\textbf{Für einen Fahrradfahrer ist es möglich, mit durchschnittlicher Geschwindigkeit in 90\% der Fälle eine ,,Grüne Welle'' zu haben, in der er um die Binnen- und Außenalster fährt.}
Dadurch, dass in einer durchschnittlichen Arbeitswoche Fahrradfahrer von und zu der Arbeit fahren, sind zwei Fahrten um die Binnen- und Außenalster vorgesehen und zu schaffen, um die größtmögliche Menge an Fahrradfahrern abzudecken.

\textbf{Die Änderung von Lichtsignalschaltzeiten wirkt sich auf die Möglichkeit einer für Fahrradfahrer erreichbaren, ,,Grünen Welle'' aus.}
Trotz hohen Verkehrs können Fahrradfahrer bei einer bestimmten Lichtsignalschaltung eine ,,Grüne Welle`` erreichen, sollte die Verkehrslast dafür nicht zu hoch sein und die Schaltung genügend Freiraum für die zurückgelegte Distanz lassen.

\textbf{Die Veränderung der Verkehrslast wirkt sich auf die Möglichkeit einer für Fahrradfahrer erreichbaren, ,,Grünen Welle'' aus.}
Sobald im Verkehr eine zu große Menge an Personenkraftwagen, Fußgängern oder anderen Fahrradfahrern vorliegt, wird die Wahrscheinlichkeit einer ,,Grünen Welle'' immer geringer, da diese den Verkehr zu stark aufhalten und damit potenziell Staus verursachen können.
