% The sixth chapter, containing a summary of this paper and the outlook for future papers
% @author Kalvin Döge
%


\chapter{Zusammenfassung und Ausblick zukünftiger Forschungsarbeiten}\label{ch:summary}

Diese Arbeit beschäftigte sich mit einigen Interaktionsfacetten des Verkehrs, der Simulation innerhalb einer digitalen Zwillingsstadt und die experimentelle Bestimmung einer passenden Schaltung für Lichtsignalanlagen.
Auf dem Weg zur Lösungsentwicklung sind dabei eine Reihe von Schwierigkeiten eingetroffen und den Werdegang dieser Arbeit leider erschwerten.
Der Umgang mit diesen Problemstellen hat aber dafür immer noch zu einem Ergebnis geführt und vor allem dem Author dieser Arbeit einiges gezeigt:

Das MARS-Framework bietet geeignetes Simulationspotenzial, welches nicht nur auf SmartOpenHamburg beschränkt ist.
Dabei wurde dem Author in mehreren Wegen gezeigt, unter anderem durch die von MARS implementierte und verfügbare Skalierbarkeit des Ortes und der Agentenanzahl, die Integration von Echtzeitsensordaten und auch die stetige Erweiterung des digitalen Zwillings.
Auch wenn dabei der Erfahrungsgrad von ihm anfänglich gering war, so ist er im Nachhinein betrachtet informierter als zuvor und wäre einige Aspekte dieser Arbeit anders angegangen.

Trotz all dem ist bei dieser Arbeit gezeigt worden, dass mit den simulierten Experimenten eine allumfassende Zeitschaltung nicht der geeignete Lösungsansatz für ,,grüne Wellen`` ist, auch wenn sich eine finden ließ.
Stattdessen sollen Städte lieber in nachhaltigere, ressourcenarme Alternativen investieren, die bereits in anderen Forschungsbeiträgen als solche untersucht wurden: Erweiterungen von Fahrradwegen, Fahrradverkehr an Ampeln zu bevorzugen, Steuerung der Schaltungen durch künstliche Intelligenzen, Verleihstationen von Pkws und Fahrrädern weiter verbreiten und noch einiges mehr.

Ebenso gibt diese Arbeit einen Ausblick für zukünftige Forschung, als dass folgende Verbesserungen oder Fortsetzungen untersucht werden könnten:
An das entwickelte Modell könnten Echtzeitdaten vom Verkehr eingespeist werden, die es an manchen Lichtsignalanlagen bereits gibt.
Diese Daten umfassen manchmal nicht nur den Verkehr, sondern auch die aktuellen, technischen Daten der Lichtsignalanlagen selbst.
Damit wären akkuratere Modelle möglich und mit einer Analyse des Verhaltens bestünde die Möglichkeit, eine künstliche Intelligenz zu trainieren.
Außerdem könnten Lichtsignalanlagen selbst noch erweitert werden, denn es gibt verschiedene Ampeltypen, die im Verkehr zum Einsatz kommen.
Die Detailtiefe ließe sich dabei sogar noch mehr erweitern, wenn man die Vorschriften für Lichtsignalanlagen von der Stadt Hamburg erhalten könnte.
Beispielsweise würde darunter fallen: Welche Sicherheitsvorkehrungen an Kreuzungen und welche an einfachen Straßen gelten, welche Vorkehrungen bei Rechtsabbiegern gelten, wie die Sicherheit bei Fahrradampeln gewährleistet wird und so weiter.

Das Fortsetzen der Arbeit mit diesen Aspekten könnte dem Forschungsaspekt weitere Tiefe geben, die dann entsprechend zu besseren oder mehr Erkenntnissen führen.
