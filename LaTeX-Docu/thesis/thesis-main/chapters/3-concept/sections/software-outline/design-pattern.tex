% The sub section about the simulations design pattern
% @author Kalvin Döge
%

\subsection{Entwurfsmuster im System}\label{subsec:design-pattern}

Damit die Umsetzung im Implementationskapitel mancher Konzepte ersichtlich wird, wird im Folgenden noch auf die genutzten Entwurfsmuster des Systems eingegangen:

\textbf{Agent-Entität-Umwelt-Muster:}

Wie im fachlichen Datenmodell~\ref{fig:class-diagramm} erwähnt, basiert diese Simulation auf dem Agent-Entität-Umwelt-Muster.
Dabei wird eine Simulationsebene bereitgestellt, die Umwelt, die in dieser Arbeit der \code{HumanTravelerLayer} und der \code{TrafficLightLayer} ausmachen.
Für die Agenten \code{HumanTraveler} und \code{BicycleLeader} sind die beiden Layer die Umwelt, in der sie sich bewegen und mit anderen Agenten sowie Entitäten, zum Beispiel den \code{TrafficLight}s, interagieren können.
Agenten können selbst entscheiden, wie sie ihr Ziel erreichen, während sie dabei nur an die Regeln ihres Handels gebunden sind.
In dem Fall dieser Arbeit wollen \code{HumanTraveler} zu einem zufällig festgelegten Ziel, suchen sich eine Route und nutzen dann die selbst ausgewählte Modalität zum Erreichen dieses Ziels.
Die Entitäten sind in dem Rahmen dieser Arbeit die Lichtsignalanlagen, die \code{TrafficLight}s, aber ebenso auch die Fahrräder, Pkws oder mietbaren Äquivalente.
Entitäten selbst sind nicht in der Lage, eigene Entscheidungen zu treffen und zu ,,handeln``.
Pkws oder Fahrräder selbst lassen sich steuern, sind dabei aber immer gleich in ihrem Verhalten:
Wird die Bremse eines Fahrrads betätigt, so wird das Fahrrad entschleunigen.
Die Entscheidung zu Bremsen kommt aber von dem Agenten, zum Beispiel dem \code{BicycleLeader}.

\textbf{Factory-Pattern zur Agentenerstellung in der Simulationsumgebung:}

Für die Erstellung der Agenten und Entitäten innerhalb der Umwelt wird je ein Factory-Pattern eingesetzt, dass die Erstellung über Parameter vereinfacht.
Die benötigten Daten werden ausgelesen, in der ,,Factory`` verarbeitet und ein neuer Agent oder eine neue Entität wird zurückgegeben, mit den eingegebenen Eigenschaften.
Das Factory-Pattern ermöglicht es, eine zentrale Stelle der Erstellung zu haben, bei der mit sich ändernden Parametern dennoch im Kern ähnliches dabei herauskommt.
Beispielsweise werden \code{HumanTraveler} sich von den Daten her untereinander unterscheiden: Manche nutzen ein Fahrrad, manche ein Pkw, was dennoch unter demselben Punkt zusammengefasst wird: ,,Vehikel``.
