% This subsection is about the technical and non technical requirements of the system
% @author Kalvin Döge
%

\subsection{Funktionale und nichtfunktionale Anforderungen}\label{subsec:requirements}

Im Folgenden wird auf die Aufgaben des entwickelten Systems eingegangen, die es einzuhalten hat.

\textbf{Funktionale Anforderungen:}

\begin{itemize}
    \item \textbf{Digitaler Zwilling als Simulationsumgebung:} Das System soll als Infrastrukturreferenz für die Simulation die Stadt Hamburg, die Straßen und ihre Lichtsignale nehmen.
    \item \textbf{Simulation eines ganzen Tages:} Das System soll für die Simulation die Zeit eines Wochentags von 0 Uhr morgens bis 23:59 Uhr simulieren.
    \item \textbf{Einbindung der HumanTraveler:} Das System bindet die vorher berechneten, aktiven HumanTraveler in die Simulation ein.
    \item \textbf{Zielsetzung der HumanTraveler:} Das System lässt jeden HumanTraveler ein Ziel haben, das sie mit beliebigen Modalitäten erreichen wollen.
    \item \textbf{Einbindung des BicycleLeaders:} Das System erstellt für jede Stunde einen BicycleLeader.
    \item \textbf{Zielsetzung des BicycleLeaders:} Das System lässt den BicycleLeader die acht Punkte der abzufahrenden Route nacheinander als Ziel haben.
    \item \textbf{Einbindung der HumanTraveler:} Das System bindet die relevanten Lichtsignalanlagen an den zugehörigen Straßen, Spuren und Kreuzungen in die Simulation ein.
    \item \textbf{Anhalten der HumanTraveler:} Das System soll die HumanTraveler im System an den Lichtsignalanlagen für die Rotphasen anhalten und für die Grün- sowie Gelbphasen weiterfahren lassen.
    \item \textbf{Anhalten des BicycleLeaders:} Das System soll das Simulieren des BicycleLeaders stoppen, sofern dieser bei einer Lichtsignalanlage hält.
    \item \textbf{Ausgabedaten der HumanTraveler:} Das System soll die gefahrenen Strecken der HumanTraveler nach deren Simulierung ausgeben.
    \item \textbf{Abbruchausgabedaten des BicycleLeaders:} Das System soll beim Anhalten des BicycleLeaders die gefahrene Strecke, den aktuellen Zeitpunkt, die Position der Lichtsignalanlage und die an der Lichtsignalanlage wartenden Agentenanzahl ausgeben.
    \item \textbf{Erfolgsausgabe des BicycleLeaders:} Das System soll beim Erreichen des Endziels von dem BicycleLeader den aktuellen Zeitpunkt sowie das erfolgreiche Ankommen ausgeben.
\end{itemize}

Zu den funktionalen Anforderungen gehören noch die nicht-funktionalen:

\textbf{Nicht-Funktionale Anforderungen:}

\begin{itemize}
    \item \textbf{Grüne Welle für 90\% der Fahrten:} Das System soll so lange weiter simulieren, bis eine Lichtsignalschaltung gefunden wurde, die dem BicycleLeader in mindestens 90\% der Fälle eine ,,grüne Welle`` ermöglicht.~Bei 24 Einträgen pro Simulationsdurchgang sind das 3 oder weniger Einträge, die nicht eine ,,grüne Welle`` sein dürfen.
    \item \textbf{10 Simulationen pro Szenario:} Das System soll für jede Änderung der Lichtsignalschaltung mindestens zehn Simulationen durchführen.
    \item \textbf{Fahrtwege als GeoJSON:} Das System soll die Ausgabedaten der Fahrtwege als GeoJSON ausgeben, sei es von dem BicycleLeader oder von dem HumanTraveler.
    \item \textbf{Abbruchsausgabedaten des BicycleLeaders als CSV:} Das System soll die Ausgabedaten beim Simulierungsabbruch des BicycleLeaders in einer CSV hinterlegen.
\end{itemize}
