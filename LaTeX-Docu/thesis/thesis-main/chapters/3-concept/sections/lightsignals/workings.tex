% The inner workings of the light signals
% @author Kalvin Döge
%

\subsection{Funktionsweise von Lichtsignalschaltungen}\label{subsec:workings}

Die Funktionsweise der Lichtsignalschaltungen während der Simulation lässt sich in zwei Schritte zusammenfassen: 1.\ Timer fortsetzen und 2.\ die Warteschlange überprüfen.

Der Timer wird im ersten Schritt um eine Sekunde fortgesetzt.
Ist der aktuelle Timer-Wert über der Zeitgrenze von einer Rot-, Gelb- oder Grünphase, wird die Phase gewechselt.
Ist der Timer-Wert über die Summe von Rot-, Gelb- und Grünphasenlänge hinaus, setzt sich der Timer wieder zurück auf 1 verstrichene Sekunde.

Im zweiten Schritt wird die Warteschlange überprüft.
Ist der Verkehrsteilnehmer nicht mehr an dieser Ampel, weil der Teilnehmer der erste in der Schlange ist und die derzeitige Ampelphase Grün ist, so wird dieser aus der Warteschlange entfernt.
Das Wegfahren selbst unternimmt immer noch der Pkw-Fahrer selbst.
