% The section about the agents' modalities of the simulation itself
% @author Kalvin Döge
%

\subsection{Voraussetzungen}\label{subsec:prerequisities}

Es gibt zwei Arten von Agenten, die in der Simulation agieren: Den Nebenagenten, fortan ,,\code{HumanTraveler}`` genannt, und dem Hauptagenten, fortan ,,\code{BicycleLeader}`` genannt.

Der \code{BicycleLeader} ist der Fokus dieser Arbeit.
Dieser wird immer mit einem eigenen Fahrrad oder einem mietbaren Fahrrad ausgestattet, mit dem er vorgegebene Punkte auf einer Route um die Alster abfahren soll.
Sein Ziel ist es, eine Runde um die Alster zu fahren, ohne dabei auf 0 km/h bremsen zu müssen.
Langsames Fahren ist hier nicht mit einbezogen als Fehlschlagbedingung, da eine Schwelle für ,,zu langsames`` Fahren, sodass der \code{BicycleLeader} sein Gleichgewicht nicht mehr halten könne, von einer Reihe von Faktoren abhängt, die außerhalb des Rahmens dieser Arbeit wären: das Alter des BicycleLeaders, die Erfahrung mit dem Fahrrad, das angestrebte Fahrverhalten, Höhenprofile der Umgebung, Wetterbedingungen und noch einige Aspekte mehr.

\code{HumanTraveler} sind dabei die Einwohner, die mit Lichtsignalschaltungen interagieren und auf den Straßen dem \code{BicycleLeader} in die Quere kommen.
Das Ziel der \code{HumanTraveler} ist es lediglich, ein zufällig zugewiesenes Ziel um die Alster herum zu erreichen, bevor sie aus der Simulation entfernt werden.
