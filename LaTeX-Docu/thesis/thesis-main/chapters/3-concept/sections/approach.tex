% The section about the approach of this project on how to determine the results
% @author Kalvin Döge
%


\section{Herangehensweise zur Ergebnisermittlung}\label{sec:approach}

Um die optimale Lichtsignalschaltung zu ermitteln, wird experimentell vorgegangen und die Lichtsignalphasen angenähert.
Mit den gegebenen Daten aus den vorherigen Kapiteln lässt sich eine Ober- und Untergrenze etablieren.
Die Obergrenze stellt hierbei den ungenügenden aber der Realität entsprechenden Zustand dar, der zu verbessern gilt.
Dieser ist nach den Messungen an der Lichtsignalanlage mit 35 Sekunden für die Grün-, 1 Sekunde für die Gelb- und 40 Sekunden für die Rotphase ausgestattet.

Fortan wird das Format für die Phasenlängen zwischen Grün, Gelb und Rot in diesem Format angegeben, mit der eben genannten Obergrenze als Beispiel: \code{35 g | 40 r}.
Die Gelbphase wird nicht mit angegeben oder verändert, da sie von ihrem Nutzen her dem grünen Lichtsignal ähnelt und Fahrzeugen erlaubt, weiterzufahren.
Entsprechend wird sich nur auf die Veränderung der Grün- und Rotphase in dieser Arbeit beschränkt.

Die Untergrenze ist festgelegt als \code{35 g | 0 r}.
Dieser Fall wäre nur in einer perfekten, realen Welt nützlich, in der bei Kreuzungen Fahrzeuge autonom sich sortieren und Straßen befahren.
Dadurch, dass heutzutage Staus und Unfälle mitbekommen werden, ist diese Schaltung nicht wirklich praktikabel, also stellt sie damit die untere Grenze dar.
Technisch hat die Länge der Grünphase bei einer Rotphase von 0 Sekunden keine Relevanz mehr, da es in der Simulation nie zu Rot beziehungsweise der Abbruchbedingung für die \code{BicycleLeader} kommen kann, weshalb der Einfachheit halber der Wert der Obergrenze übernommen wurde.


Bei der Annäherung selbst einer geeigneten Schaltung wird dabei antiproportional von der Obergrenze ausgegangen:
Ist die Obergrenze mit \code{35 g | 40 r} nicht ausreichend in der Erfolgsrate, so wird die Grünphasenlänge dann verdoppelt und die der Rotphase halbiert: \code{70 g | 20 r}.
Ist die Annäherung nun aber ausreichend oder durchgängig positiv, so wird wieder in die andere Richtung verlängert und verkürzt.
Diesmal wird aber nicht die Grünphasenlänge halbiert und die Rotphasenlänge verdoppelt, da dies wieder zu der Obergrenze führen würde: \code{35 g | 40 r} .
Stattdessen wird die Rotphase mit 1.5 beziehungsweise 3/2 und die Grünphase mit 2/3 multipliziert, wie im folgendem Dreisatz zu erkennen ist:

\begin{align}
    40~\unit{r}~&\widehat{=}~35~\unit{g} \\
    1~\unit{r}~&\widehat{=}~1.400~\unit{g} \\
    30~\unit{r}~&\widehat{=}~46.\overline{6}~\unit{g} \\
\end{align}
\label{fig:dreisatz}

Mit dem Ergebnis aus dem Dreisatz~\ref{fig:dreisatz} liegen dann die neuen Phasenlängen vor: \code{47 g | 30 r}.
Wichtig dabei ist zu beachten, dass Sekunden die kleinste Zeiteinheit in der Simulation sind, weshalb auf ganze Zahlen gerundet werden muss.
