% introductory chapter
% @author Kalvin Döge
%
\chapter{Einleitung}\label{ch:einleitung}

Mit dem Klimawandel wird es immer deutlicher, dass man selbst lieber auf den Pkw verzichten und auf den öffentlichen Personennahverkehr oder das Fahrrad wechseln sollte.
Doch leider ist es nicht ganz so einfach: Gerade in der Großstadt Hamburg gehört Stau tagsüber quasi zum Alltag auf den Straßen.
Pkws, Lkws und Fahrradfahrer teilen sich auf manchen Wegen immer wieder dieselbe Spur und halten auch an derselben Ampel, wenn wieder keine Fahrradspur gerade zur Verfügung steht.
Doch die Stadt Hamburg fängt an, das Fahrrad im Verkehr in manchen Stadtvierteln zu bevorzugen und die Lichtsignalschaltungen nach ihnen auszurichten~\cite{NDR2022}.
So auch gegen das Ende letzten Jahres: Im Stadtteil Eimsbüttel sollen Fahrradfahrern und Fußgängern dem Autoverkehr bevorzugt werden, indem eine Ampel an einer stark befahrenen Überquerung vor anderen vorlassen.

Auch wenn dies nur ein Anfang ist, so lässt sich daraus die Frage ableiten: Ließen sich Lichtsignalschaltungen so zeitlich einstellen, dass sie einem Fahrradfahrer den gesamten Weg eine ,,Grüne Welle`` geben, noch bevor man erst an der Ampel einen Knopf betätigen muss?

In dieser Arbeit soll diese Frage bei einer verkehrslastigen Rundfahrt untersucht werden und eine mögliche Zeitschaltung bereithalten können.

%
% @author Kalvin Döge
%
\section{Struktur der Bachelorarbeit}\label{sec:struktur-der-bachelorarbeit}

Diese Arbeit teilt sich, nach diesem ersten Kapitel ,,Einleitung'', in fünf weitere Kapitel auf:

\begin{itemize}

    \item Dem zweiten Kapitel ,,Methodik'', welches sich mit Erklärungen zum eigentlichen Vorgehen beschäftigt, mit der die Hypothesen vorgestellt werden.
    \item Dem dritten Kapitel ,,Konzept'', welches die Implementation textuell vorbereitet und das Simulationsmodell, als auch die Ein- und Ausgabedaten beschreibt und erklärt.
    \item Dem vierten Kapitel ,,Implementation'', welches sich mit dem Quellcode und dessen besonderen Aspekten widmet.
    \item Dem fünften Kapitel ,,Evaluation'', welches die Ausgabedaten in Bezug zu den Eingabedaten und Modelleinstellungen erläutert und darstellt, als auch den Bezug zu den Forschungsfragen und Hypothesen wieder herstellt.
    \item Dem sechsten und letzten Kapitel ,,Zusammenfassung'', in dem die Ergebnisse der Forschungsarbeit genannt und übersichtlich nochmal aufgeführt sind.

\end{itemize}

%
% @author Kalvin Döge
%
\section{Inhalt der Arbeit}

Zur Bestimmung einer ,,Grünen Welle'' für Fahrradfahrer um die Binnen- und Außenalster, kommen folgende Fragen zur Simulation des Szenarios auf:

\begin{itemize}
    \item Wie sieht ein durchschnittlicher Fahrradfahrer, Fußgänger und ein Personenkraftwagen in dem Modell aus?
    \item Welche Strecke fährt der Fahrradfahrer um die Binnen- und Außenalster, um eine Rundfahrt unternommen zu haben?
    \item Was für Eigenschaften muss die Lichtsignalschaltung in dem Modell haben, damit sie geeignet für eine ,,Grüne Welle`` ist?
    \item Wie stark wirkt sich die Auslastung der Straßen auf das Lichtsignalnetz aus,
    \begin{itemize}
        \item[-] wenn zu verschiedenen Uhrzeiten am Tag die Alsterrundfahrt unternommen wird?
        \item[-] wenn die Anzahl an Verkehrsteilnehmern, also Personenkraftwagen, Fußgänger und Fahrradfahrer, erhöht oder gesenkt wird?
    \end{itemize}
    \item Ist es überhaupt möglich, dass Fahrradfahrer in mindestens 90\% der Fälle, die er um die Alster fährt, eine ,,Grüne Welle'' erhält, ohne anzuhalten?
    \item Wie müssen die Lichtsignalzeiten angepasst werden, wenn der Verkehr sich zu stark auf die ,,Grüne Welle'' auswirkt?
\end{itemize}

Um die Aspekte genauer zu untersuchen, lassen sich aus ihnen forschungsrelevante Hypothesen aufstellen, die im Folgenden genauer definiert werden:

\textbf{Für einen Fahrradfahrer ist es möglich, mit durchschnittlicher Geschwindigkeit mindestens zweimal pro Wochentag um die Binnen- und Außenalster zu fahren und dabei eine ,,Grüne Welle'' zu haben.}
Dadurch, dass in einer durchschnittlichen Arbeitswoche Fahrradfahrer von und zu der Arbeit fahren, sind zwei Fahrten um die Binnen- und Außenalster vorgesehen und zu schaffen, um die größtmögliche Menge an Fahrradfahrern abzudecken.

\textbf{Die Änderung von Lichtsignalschaltzeiten wirkt sich auf die Möglichkeit einer für Fahrradfahrer erreichbaren, ,,Grünen Welle'' aus.}
Trotz hohen Verkehrs können Fahrradfahrer bei einer bestimmten Lichtsignalschaltung eine ,,Grüne Welle`` erreichen, sollte die Verkehrslast dafür nicht zu hoch sein und die Schaltung genügend Freiraum für die zurückgelegte Distanz lassen.

\textbf{Die Veränderung der Verkehrslast wirkt sich auf die Möglichkeit einer für Fahrradfahrer erreichbaren, ,,Grünen Welle'' aus.}
Sobald im Verkehr eine zu große Menge an Personenkraftwagen, Fußgängern oder anderen Fahrradfahrern vorliegt, wird die Wahrscheinlichkeit einer ,,Grünen Welle'' immer geringer, da diese den Verkehr zu stark aufhalten und damit potenziell Staus verursachen können.

%
% @author Kalvin Döge
%


\section{Fokus der Arbeit}\label{sec:focus-of-thesis}

Diese Arbeit beschäftigt sich mit der Bestimmung einer Zeitschaltung für Lichtsignalanlagen, die für alle Anlagen gleichermaßen gelten soll und mit einer hohen Wahrscheinlichkeit eine ,,grüne Welle`` aus der Sicht der Agenten bewirken soll.
Dabei wird das MARS-Framework mit SmartOpenHamburg verwendet, um die Binnen- und Außenalster als Simulationsgebiet und Fokuspunkt des Verkehrs möglichst genau nachzustellen.
Der Nutzen für die Bestimmung einer festen Zeitschaltung ist die einfache Änderung bestehender Lichtsignalanlagen in der echten Welt: Innerhalb von Hamburg gibt es mehrere hundert Lichtsignalanlagen, mit 1260 Anlagen bereits innerhalb des simulierten Bereiches.
Das stetige Anpassen der Signalanlagen durch Verlängern oder Verkürzen von Phasen über, zum Beispiel, einer künstlichen Intelligenz hat nicht nur die typischen Synchronisationsprobleme eines verteilten Systems als Schwierigkeit, sondern auch die Technik als Problem.
Die Verkehrszentralstellen sind potenziell nicht ausgestattet für eine gut ausgearbeitete, künstliche Intelligenz oder können nur an eine begrenzte Menge von Anlagen häufige Phasenänderungen zusenden.
Stattdessen ist ein einmaliges Einstellen aller Anlagen mit einer bestimmten Grün-, Gelb- und Rotphasenlänge lediglich eine Abänderung der Phasenlängen und benötigt weder Synchronisation noch stabile Kommunikationswege zu den Anlagen.

Außerdem sind bisherige Forschungsarbeiten bei in Reihe geschalteten, ,,grünen Wellen`` stets nur auf Vehikel wie Pkws angeschaut worden, die keine flexiblen Ausweichmöglichkeiten wie ein Fahrrad einbeziehen in die Agentensimulation.
Fahrräder können stets eine neue Route einschlagen oder, wenn plötzlich ein Fahrradweg aufkommt, auf diese wechseln und eine Lichtsignalanlage später oder früher erreichen.
Ebenso haben Fahrräder nur auf Fahrradwegen die Sicherheit, dort fahren zu dürfen, was sie bei Straßen in dem Verkehr stark benachteiligt.
Dadurch ist es erkennbar, dass sie teilweise auf anderen Routen als Autos fahren müssen und damit eine vorgeplante ,,grüne Welle`` über ein Straßennetzwerk potenziell nicht einhalten können.
Spätestens dann ist eine Bindung der Lichtsignalphasen an die Distanz zu vorherliegenden Lichtsignalanlagen nicht mehr so nützlich und lässt den Fahrradfahrer anhalten.

Entsprechend ist eine zeitlich festgelegte Lichtsignalschaltung bei allen Anlagen ein wirtschaftlicher und technisch einfacherer Lösungsweg, den es in dieser Arbeit zu ermitteln gibt.

