%
% @author Kalvin Döge
%
\section{Simulationsmodelle}\label{sec:simulationsmodelle}

Um den Verkehr und deren Teilnehmer um die Binnen- und Außenalster so einstellbar wie möglich zu machen, empfehlen sich zwei Arten einer Modellsimulation:

\begin{itemize}

    \item Eine agentenbasierte Simulation
    \item Eine ereignisgesteuerte Simulation

\end{itemize}

\textbf{Agentenbasierte Simulation}
In dieser Simulationsart werden Akteure in eine Umwelt gebracht, die in Echtzeit sowohl mit sich gegenseitig, als auch mit der Umgebung interagieren und mithilfe eines festgelegten Regelsatzes Entscheidungen treffen, die andere Agenten oder die Umwelt beeinflussen können~\cite{Baldwin2010}.
Wesentlich ist dabei auch, dass die Akteure selbst Entscheidungen treffen können und ein eigenes Verhalten aufweisen.
In einer Simulation mit Lichtsignalschaltungen und die Bestimmung einer optimalen ,,Grünen Welle'' könnten die Akteure zum Beispiel die Fußgänger, die Fahrradfahrer und die Personenkraftwagen darstellen, während die Umwelt das Straßennetz mit den Lichtsignalen wäre.
Personenkraftwagen könnten gleichzeitig mit Fahrradfahrern interagieren, indem sie zum Beispiel Plätze vor Ampeln wegnehmen oder Staus verursachen, während Fußgänger zum Beispiel die Ampeln betätigen und damit Fahrradfahrern die ,,Grüne Welle'' verhindern könnten.

\textbf{Ereignisgesteuerte Simulation}
Neben dem agentenbasierten Modell, gibt es ebenso auch die Möglichkeit einer ereignisgesteuerten Simulation.
Diese Art der Simulation ist nicht in Echtzeit, hat aber wiederum im allgemeinen Ereignisse, die nacheinander passieren und abgegangen werden~\cite{Baldwin2010}.
Auf dem Weg von einem Ereignis zu einem anderen können äußere Einflüsse den Übergang zum nächsten Ereignis verändern.
Dadurch, dass aber der Wechsel von einem Ereignis zum anderen im Vornherein bei der Implementierung bekannt ist, kann man die Übergänge der Ereignisse zum Beispiel mit Zeit bereits errechnen.
